% !TEX root = ../../main.tex

\chapter{Approach To Problems}\label{ch:approach}
    \section{In Depth Definition Of Task}
    \section{Solution Form}
The sequence archetype will use the Cycler() player for our strategy each time, only editing the input to improve our score. To this model we can apply an optimised input of length n (tbd) to the player, this sequence, as per the design of the strategy will then be repeated until the games end(if n = len(game) then we are just calculating the sequence for the whole game).\\

The input sequence itself will be created using a genetic optimisation. Starting with a set of randomly generated sequences, we will have each one play the opponent and return with a score. These sequences will be ranked and the lowest x\% will be discarded, resulting in a fitter, but smaller, population than before. This smaller population will then create offspring using a |X TBD method X| pairing algorithm before mutating with |X TBD method X|. This new set of offspring will be included in the next scoring round and the process repeats for k number of rounds\\

This sequence of Play-Rank-Create-LOOP will be the basis of creating the optimal strategy for each other opponent.
    \section{initial research}

local vs global maximum
simple\footnote{ones we can figure out optimal} vs non-simple 
consecutive difference (np.diff())

    Here i will look at the algorthem in a general manner. i hope to answer questions like:
    \begin{itemize}
        \item what happens to the convergence rate to the optimal sequence for an oppenent when the initial starting poulation for our algorithm changes? 
        \item what happens to the convergence rate to the optimal sequence for an oppenent when the mutation rate of each of our population is changed?
        \item  what happens to the convergence rate to the optimal sequence for an oppenent when we change the crossover algorithm to be more random?
        \item 
    \end{itemize}
    
