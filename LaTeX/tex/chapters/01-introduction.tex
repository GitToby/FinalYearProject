% !TEX root = ../main.tex  

\chapter{Introduction}\label{ch:intro}

General discussion on what game theory is and what the PD is.
What this topic is and how it fits into the big picture.\\

\section{Iterated Prisoners Dilemma}\label{sec:iteratedPrisonersDilemma}
The Prisoners Dilemma is a classic game theory topic\ldots
Its important to game theory because\ldots
Creating general sequences to strategies is important because\ldots
Models in the real world that follow certain strategies\ldots
How we can leverage these to peruse goals.

The Prisoners Dilemma is a well known game theory problem based on the example of a pair of prisoners and their subsequent interrogation.
The game is as follows:\\

Something about the PD \\

The single game itself is very basic and is modelled in the following way: \\

\[give-a-model-here\]

The Iterated Prisoners Dilemma is the iterated version of the Prisoners Dilemma\footnote{reference this stuff dude, come on..}.
The iteration of the game is what makes the game an interesting concept, as now \textbf{learn the technical stuff and
put it here!!!!!} we are able to create strategies\footnote{When referring to ourselves, we will describe our moves as a strategy.
When referring to an opponent we can use the term opponent and strategy interchangeably.} that look to gain an upper hand based on \textbf{Something here}.

\subsection{Machine Learning \& Computer Intelligence}\label{subsec:machineLearningAndcomputerIntelligence}
This section will briefly provide a background to machine learning algorithms.
This is by no means a comprehensive look into these subjects but will provide sufficient background on technical discussion later on.\\

Machine Learning is a field of computer science that has existed since the first computers have been around.
Most famously the questions posed by Alan Turing in 1950~\cite{turing1950computing} asks `can machines think', a question that has been refined and analysed to this day.
The field of computer intelligence is rich in its complexities and has recently been making breakthroughs\footnote{Google go} on this question.
Recently there has also been record levels of funding~\cite{chui2017artificial} put in to companies which operate in this field, producing results in areas that would usually seem `solved'.\\

This report will cover one of the forms of machine learning called genetic algorithms.
Techniques of machine learning can be combined and used together in many situations, the field of mathematics research is one;
as used in~\cite{chu1997genetic}.

\subsubsection{Genetic Algorithms}\label{subsubsec:geneticAlgorithms}
Genetic Algorithms fall under a branch of machine learning called feature selection.
Techniques of using genetic algorithms for generating solutions to problems typically revolve around heuristically improving members of a population who represent these solutions.
The concepts of a genetic algorithm come from nature;
like nature we create a survival of the fittest competition\footnote{need to reference Darwin?} to evaluate a population then kill off the weakest members.
After this cull we create offspring from the most successful population or introduce new members from a predefined source.
This process is then repeated until we stop, or forever in the case of nature.
Figure~\ref{fig:genericGeneticAlgoCycle} is a diagram of this cycle.\\

Put a figure of the cycle here.\label{fig:genericGeneticAlgoCycle} \\

Our implementation of a genetic algorithm is more custom and has the following steps:
\begin{itemize}
    \item Start with a set of predefined sequences \& randomly generated sequences.
    \item We will have each one play the given opponent and return with the average score per turn.
    \item These sequences will be ranked and the lowest pairings \% will be discarded, resulting in a fitter, but smaller, population than before.
    \item This smaller population will then create offspring using a crossover pairing algorithm before mutating with a selection of.
    \item This new set of offspring will be included in the next scoring round and the process repeats for k number of rounds.
\end{itemize}

Put a figure of the cycle here.\label{fig:geneticAlgoCycle} \\

Initially we create a heuristic function, say our fitness function, which is a measure of how successful a candidate in our population is.
Then we run our whole population through this function, ranking each one by how successful their score is.
At this point we can create a cut off\footnote{Can often be referred to as the bottleneck} to decide which of the population not to put through to the next round.\\

\subsubsection{Bayesian Optimization}

\section{Brief Overview}\label{sec:briefOverview}
In this document I will be be looking at the creation of sequences to beat given players in The Iterated Prisoners Dilemma\footnote{Reference this for some background}.
My research looked into just the single opponent use case, but the idea of designing a sequence for a given number of opponents is looked at in the further study of the report.
This task is the


Problem:\\
\begin{quotation}
    Given a certain opponent, \(O\), (with a provided strategy, S) what is the best possible sequence of moves, in a game of n turns, made by my strategy to maximise my players score?
\end{quotation}

