% !TEX root = ../../../main.tex  

\chapter{Introduction}\label{ch:intro}

General discussion on what game theory is and what the PD is. what this topic is and how it fits into the big picture.
The Prisoners Dilemma is a classic game theory topic... Its important to game theory because... creating general sequences to strategies is important because... models in the real world that follow certain strategies... how we can leverage these to peruse goals.

    \section{Background}
        fill this section with background on game theory in when I do the course?
        \subsection{Iterated Prisoners Dilemma}\label{ssec:IPD}
        The Prisoners Dilemma is a well known game theory problem based on the example of a pair of prisoners and their subsequent interrogation. The game is as follows:\\ 

        Something about the PD \\ 

        The single game itself is very basic and is modeled in the following way: \\ 

        $$give-a-model-here$$

        The Iterated Prisoners Dilemma is the iterated version of the Prisoners Dilemma\footnote{reference this stuff dude, come on..}. The iteration of the game is what makes the game an interesting concept, as now \textbf{learn the technical stuff and put it here!!!!!} we are able to create strategies\footnote{When referring to ourselves, we will describe our moves as a strategy. When referring to an opponent we can use the term opponent and strategy interchangeably.} that look to gain an upper hand based on \textbf{Something here}

        
        \subsection{Machine Learning Concepts}
        This section will briefly provide a background to machine learning algorithms implemented in the axelrod-dojo\footnote{referencing opportunity here}. This is by no means a comprehensive look into these subjects but will provide sufficient background on technical discussion later on.
            \subsubsection{Genetic Algorithms}
            % rephrase this section
            Genetic Algorithms are a description of techniques for generating solutions complex problems such as searching and, in our case, optimization\footnote{Mitchell, Melanie (1996). An Introduction to Genetic Algorithms. Cambridge, MA: MIT Press. ISBN 9780585030944. learn to reference soon}. The basis of a genetic algorithm is focused on a cycle of evolution. Like nature, we create a survival of the fittest concept\footnote{need to reference Darwin?} to evaluate a population, kill off the weakest members and create offspring from the most successful population.\\
            
            Put a figure of the cycle here. \label{fig:genetic algo cycle} \\
                       
            Initially we create a heuristic function, say our fitness function, which is a measure of how successful a candidate in our population is. Then we run our whole population through this function, ranking each one by how successful their score is. At this point we can create a cut off\footnote{Can often be referred to as the bottleneck} to decide which of the population not to put through to the next round.\\ 

            

            \subsubsection{Bayesian Optimization}

    \section{Brief Overview}
    In this document I will be be looking at the creation of sequences to beat given players in The Iterated Prisoners Dilemma\footnote{Reference this for some background}. My research looked into just the single opponent use case, but the idea of designing a sequence for a given number of opponents is looked at in the further study of the report. This task is the 
    
    
    Problem:\\
    Given a certain opponent, $O$, (with a provided strategy, S) what is the best possible sequence of moves, in a game of n turns, made by my strategy to maximise my players score? 

