% !TEX root = ../main.tex

\chapter{Literature Review}\label{ch:literature}
\section{Background}\label{sec:background}
The Prisoners Dilemma a large area of repeated games in Game Theory.
This is due to the game being a good example of strategies that give a cooperative benefit to repeated games.
It has been used to describe actions of people and governments in situations stretching from warfare~\cite{tooby1988war,aumann1992handbook}, finance~\cite{cable1997finance} and politics~\cite{snidal1985Politics} to sex and relationships~\cite{low2015sex}.
Because of its applicability to real world scenarios there is a strong desire to understand how strategies are beaten to either exploit flaws or counter opponents in real world scenarios. \\

The Prisoners dilemma became a large research area in the combination of mathematics and computer science after Robert Axelrod published his work named "Effective Choice In The Prisoners Dilemma"~\cite{axelrod1980effective}. 
In it he makes an introduction to how tournaments are run and the properties of successful results.
His method of experimenting became the standard for handling the IDP problem.
After the tournament is complete he describes what successful strategies had in common; 
It turns out the majority of strategies have properties of niceness and forgiveness.
This allowed them to thrive in the tournament and have overall scores that rose above strategies without niceness or forgiveness. \\

Since Axelrods original tournament there have been many research papers on what makes a successful strategy for a specific objective.
For example~\cite{press2012iterated} looks at how to remove an opponents moves to reach a high score and ~\cite{mittal2009optimal} looks at a range of different objectives at once.
These specific objectives are really the core part of applying Game Theory and The Prisoners Dilemma in the really word. 
Objectives are the wrapper for which we can work with real world scenarios, for example in a the cold war it was not the goal to get as many missiles as possible to pass an oppositions defence but to not allow any missiles through your own defence; i.e. minimising your opponents score. 
In a football tournament where winning is the number or goals your team scores it doesn't really matter how many goals you get in so long as you score the most overall.
There are also some very useful applications such as rent splitting or work asignments~\cite{goldman2015spliddit}, all of which were programmed and deployed to a web server for general use online\footnote{http://www.spliddit.org}


\section{Strategy Structures}\label{sec:stratergyStructures}
Strategies can be defined in multiple ways~\cite{harper2017reinforcement}.
Each method of representing a strategies has its benefits and drawbacks, though there is no method (or strategy in that case) that is best in all cases.
There are methods of creating strategies that remove the way opponents play in order to create overall scores that desired.
Subsection~\ref{subsec:zdExtortion} discusses this in more detail.
Ways of structuring a strategy are shown below:

\begin{itemize}
 \item LookerUpper, find examples
 \item Gambler
 \item Neural Networks
 \item Finite State Machines
 \item Hidden Markov Models
 \item Explicit Move List
 \item Mixtures
\end{itemize}

% ToDo: add figures of the above, where can I find them?

\subsection{Equivalent strategies}\label{subsec:equivalentStrategies}
When performing analysis of an opponent using a GA it can sometimes be useful to brute force the starting positions of our feature selection to create rare\footnote{For example of a random sequence that is a totality of Cs is incredibly rare, with around \(6.2^{-61}\) chance of occurring} starting points.
Looking at certain opponents there are occasions some strategies that look indistinguishable from others;
for example Alternator and Random~(0.5) can very often create the same output.
One of the ways we are able to identify strategies is the process of fingerprinting~\cite{Ashlock2004,Ashlock2008}.

Another way is to look up their structure definitions in the same model (for example\ldots~)

\section{Strategies Of Interest}\label{sec:strategiesOfInterest}
\subsection{TitForTat}\label{subsec:titfortat}
Tit For Tat was the
\subsection{Alternator}\label{subsec:alternator}
See Subsection~\ref{subsec:random}.
\subsection{Grudger}\label{subsec:grudger}
See Subsection~\ref{subsec:random}.
\subsection{Random}\label{subsec:random}

For all the strategies () given above we are able to logically deduce that the best counter is a totality of Ds, no matter what their definition structure is.
This leads us to consider what the common theme is between the strategies as to be beaten in the same way.



\subsection{EvolvedFSM16}\label{subsec:evolvedFSM16}
\subsection{CollectiveStrategy}\label{subsec:collectiveStrategy}
\subsection{ZDExtortion}\label{subsec:zdExtortion}
This is the paper~\cite{press2012iterated}
\subsection{Cycler}\label{subsec:cycler}

\section{Genetic Algorithms}
This work will focus on Genetic Algorithms (GAs) which are a specific form of Machine Learning (ML).
Advanced techniques of machine learning can be combined and used together\footnote{Techniques for teaching and versioning static algorithms such as building a `clever' game AIs, where the core concept of the AI is fine tuned using GA in an development environment but isnt implemented into the game~\cite{bakkes2009rapid}.} in many situations, lots of these techniques combine genetic algorithms or some sort of fitness testing within a larger scope.
The field of mathematics research is one which has plenty of examples of GAs in action;
for example the same techniques as we will using is used in~\cite{chu1997genetic} to solve the Generalised Assignment Problem.
Another example, \cite{bhanu1995adaptive}, used neural networks when approaching the state regulation problem. \\

In Game Theory GAs are used in a couple of areas, most prominanly in the study of the interated prisonrs dilema.