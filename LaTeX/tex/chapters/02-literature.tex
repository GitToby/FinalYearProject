% !TEX root = ../main.tex

\chapter{Literature Review}\label{ch:literature}
\section{Background}\label{sec:background}
The Prisoners Dilemma a large area of repeated games in Game Theory and has applications in the real world.
This is due to the game being a good example of strategies that give a cooperative benefit to repeated games.
It has been used to describe actions of people and governments in situations stretching from warfare~\cite{tooby1988war,aumann1992handbook}, finance~\cite{cable1997finance} and politics~\cite{snidal1985Politics} to sex and relationships~\cite{low2015sex}.
Because of its applicability to real world scenarios there is a strong desire to understand how strategies are beaten to either exploit flaws or counter opponents in real world scenarios.

The Prisoners dilemma became a large research area in the combination of mathematics and computer science after Robert Axelrod published his work named "Effective Choice In The Prisoners Dilemma"~\cite{axelrod1980effective}.
In it he makes an introduction to how tournaments are run and the properties of successful results.
His method of experimenting became the standard for handling the IDP problem.
After the tournament is complete he describes what successful strategies had in common; 
It turns out the majority of strategies have properties of niceness and forgiveness.
This allowed them to thrive in the tournament and have overall scores that rose above strategies without niceness or forgiveness.

Since Axelrods' original tournament there have been many research papers on what makes a successful strategy for a specific objective.
For example William H Press and Freeman J Dyson,~\cite{press2012iterated}, looked at how to remove an opponents moves to reach a high score and Shashi Mittal and Kalyanmoy Deb have looked at a range of different objectives at once~\cite{mittal2009optimal}.
These specific objectives are the core part of applying Game Theory and The Prisoners Dilemma in the real word and modelling situations using this field of mathematics.
Objectives are the wrapper for which we can work with real world scenarios, for example in a the cold war it was not the goal to get as many missiles as possible to pass an oppositions defence but to not allow any missiles through your own defence; i.e.minimising your opponents score.
In a football tournament where winning is the number or goals your team scores it doesn't really matter how many goals you get in so long as you score the most overall.
There are also some very useful applications such as rent splitting or work assignments~\cite{goldman2015spliddit}, all of which were programmed and deployed to a web server for general use online\footnote{http://www.spliddit.org}


\section{Strategy Structures}\label{sec:stratergyStructures}
Strategies can be defined in multiple ways~\cite{harper2017reinforcement}.
Each method of representing a strategies has its benefits and drawbacks, though there is no method (or strategy in that case) that is best in all cases.
There are methods of creating strategies that remove the way opponents play in order to create overall scores that desired.
Subsection~\ref{subsec:zdExtortion} discusses this in more detail.
Ways of structuring a strategy are shown below:

\begin{itemize}
 \item LookerUpper, for example figure~\ref{fig:tit_for_tat_LUD}.
 \item Gambler, for example the Stochastic stratergies; examples can be found in~\cite{press2012iterated}.
 \item Neural Networks.
 \item Finite State Machines, for example figure~\ref{fig:tit_for_tat_FSD}.
 \item Hidden Markov Models.
 \item Explicit Move List, for example the solutions given in Appendix~\ref{apndx:solutionGroups}.
 \item Mixtures.
\end{itemize}

% ToDo: add figures of the above, where can I find them?

\subsection{Equivalent Strategies}\label{subsec:equivalentStrategies}
Looking at certain opponents there are occasions some strategies that look indistinguishable from others;
for example Alternator and Random~(0.5) can very often create the same output.
One of the ways we are able to identify strategies is the process of fingerprinting~\cite{Ashlock2004,Ashlock2008}.
However this can be inconclusive and less accurate as desired.
An effective method of identifying equivalent strategies is to write them down in the form of the other.
For example if we can write any strategy in one of the forms given in figures~\ref{fig:tit_for_tat_FSD},\ref{fig:tit_for_tat_LUD} we know its an equivalent to Tit for Tat.

The work being completed in this paper is another method of identifying strategies; identifying the best response to them may lead to observations about how and why different strategies act in similar manners.
This wont lead to show strategy equivalence, but it will show how solution equivalence.

\begin{figure}[ht]
    \centering
    \begin{minipage}{0.48\textwidth}
        \centering
        \tikzset{EdgeStyle/.append style = {->} ,
        LabelStyle/.style = {rectangle, rounded corners, draw, fill = black!5}}
        \begin{tikzpicture}[
            startnode/.style={rectangle, rounded corners, draw, fill = blue!5},
            roundnode/.style={circle, draw=black!60, fill=green!10, very thick, minimum size=7mm}
            ]
            %Nodes
            \node[roundnode](initial_state){1}; 
            \node[startnode](start_node)[left=of initial_state]{Start};
            %Lines
            \Edge[label = $C$](start_node)(initial_state)
            \Loop[dist = 3cm, dir = NO, label = $C/C$](initial_state.north)
            \Loop[dist = 3cm, dir = SO, label = $D/D$](initial_state.south)
        \end{tikzpicture}
        \caption{Finite State diagram of strategy Tit for Tat}\label{fig:tit_for_tat_FSD}
    \end{minipage}\hfill
    \begin{minipage}{0.48\textwidth}
        \centering
        \includegraphics[width=0.6\textwidth, center]{./img/examples/tit_for_tat_LUD.pdf}
        \caption{Look Up diagram of strategy Tit for Tat}\label{fig:tit_for_tat_LUD}
    \end{minipage}
\end{figure}


\section{Strategies Of Interest}\label{sec:strategiesOfInterest}
\paragraph{Tit For Tat} was the winner of the original axelrod tournament in 1980~\cite{axelrod1980effective}. It s a very basic opponent who has strong forgiveness (it will forgive a defection after 1 move) and generosity (it will start with a cooperation) which is thought to give its well overall score in tournaments.
\paragraph{Alternator} is a `dumb' opponent, i.e. no strategic method at all. All it will do is alternate between cooperation and defection until the game ends. Playing an Alternator effectively is to defect the whole game, however identifying an alternator to play this sequence without backlash can be difficult if we're playing a potentially similar strategy such as Random or 
\paragraph{Grudger} can be considered as the most unforgiving strategy that exists.
Starting by cooperating, if you defect even once then the Grudger will defect until the end of the game.
In line with Tit For Tat, our best score will come from not `upsetting' the opponent (until the last move where it cant react.
\paragraph{Random} is the most basic stochastic opponent, and like Alternator and Cycler it is `dumb'. 
With a probability $p$ of cooperating and $1-p$ of defecting, we can just defect the whole time to beat this opponent; picking up bonus points on its cooperation moves.
\paragraph{EvolvedFSM16}  
\paragraph{CollectiveStrategy} has the property of a handshake? it will try to identify if the opponent is another
\paragraph{ZDExtortion}
This is the paper~\cite{press2012iterated}
\paragraph{Cycler}

\section{Genetic Algorithms}
This work will focus on Genetic Algorithms (GAs) which are a specific form of Machine Learning (ML).
Advanced techniques of machine learning can be combined and used together\footnote{Techniques for teaching and versioning static algorithms such as building a `clever' game AIs, where the core concept of the AI is fine tuned using GA in an development environment but isn't implemented into the game~\cite{bakkes2009rapid}.} in many situations, lots of these techniques combine genetic algorithms or some sort of fitness testing within a larger scope.
The field of mathematics research is one which has plenty of examples of GAs in action;
for example the same techniques as we will using is used in~\cite{chu1997genetic} to solve the Generalised Assignment Problem.
Another example, \cite{bhanu1995adaptive}, used neural networks when approaching the state regulation problem.

In Game Theory GAs are used in a couple of areas, most prominently in the study of the iterated prisoners dilemma.