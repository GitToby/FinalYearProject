\chapter{Developing The Codebase}\label{ch:developingthecodebase}
For the analysis a mix of Jupyter Notebooks and integrated development environments\footnote{Pycharm Professional Edition and Microsoft VS Code.} were used to write and execute code.
The main analysis was run using a factory class, written by me, called `full\_analysis'.
This class was used to wrap a query to the Axelrod-Dojo functionality and subsequently to the Axelrod Library in such a way that was easy to control batch executions.\\

The analysis itself was done using native multi-threading on a Linux OS to improve individual opponent analysis run times and the overall scalability of the project.
This will be discussed further in Subsection~\ref{subsec:settingUpAResearchEnvironment}.

\section{Libraries And External Modules}
\paragraph{Main Research Libraries}
\begin{itemize}
    \item \textbf{Axelrod} --- Used for the core of the prisoners dilemma and iterated prisoners dilemma functionality code.\cite{axelrodproject}
    \item \textbf{Axelrod-Dojo} --- Applied machine learning techniques that revolve around generating solutions to questions relating to the Axelrod library.
\end{itemize}

\paragraph{Functional libraries}
Table~\ref{table:functionalLibrares} shows the external functional libraries used, while Table~\ref{table:builtinmodules} shows the internal python builtins that where leveraged during development.
These are libraries which are not involved in the core functionality of the IPD\@.
\begin{table*}[ht]
    \centering
    \begin{tabular}{cc}
        \toprule
        Library & Reason\\
        \midrule
        \textbf{matplotlib pyplot} & For plotting graphs and images with data\\
        \textbf{pandas} & For data manipulation.\\
        \textbf{numpy} & For reducing complexity of numerical calculations.\\
        \bottomrule
    \end{tabular}
    \caption{Functional Python libraries for analysis}\label{table:functionalLibrares}
\end{table*}
\begin{table*}[ht]
    \centering
    \begin{tabular}{cc}
        \toprule
        Library & Reason\\
        \midrule
        \textbf{os} & For operating system functionality.\\
        \textbf{time} & For time calculations.\\
        \textbf{itertools} & For easier iterations over data structures.\\
        \bottomrule
    \end{tabular}
    \caption{internal builitin python modules used}\label{table:builtinmodules}
\end{table*}

\section{Setting Up A Research Environment}\label{subsec:settingUpAResearchEnvironment}
This section will go into detail on the set up and reasoning behind using specific development environments.This tutorial will also assume you're working with a local development station;
analysis on remote cloud instances of Jupyter Notebooks is possible but the set up is different.

\paragraph{Installing Basic Libraries}
Your first step should be to download and install the Anaconda distribution for your OS here: https://www.anaconda.com/download/.
This will allow you to use the integrated c++ libraries python has to offer without needing to mess about too much.
Anaconda also has them majority of Functional Libraries above and Jupyter Notebooks pre installed to make setting up much easier.
From here, follow the instructions the install wizard has to add any environment variables to allow CMD/Bash access to binaries.\\ 

Installing the Axelrod and Axelrod-Dojo libraries uses the pip tool that already comes with Anaconda and should be ready to execute after the last step.Running `pip install axelrod' then `pip install axelrod-dojo' will install these.\\

Once this is installed the `full\_analysis.py' file has to be downloaded from github: https://github.com/GitToby/FinalYearProject in the `code' directory.This can just be copied and pasted if needed all were interested in is the class to generate a sequence for an opponent.

\paragraph{Running a Test}
Figure~\ref{code:analysisExample} is some sample code that will run an analysis with the following settings
\begin{itemize}
    \item Override the default mutation frequency of \(0.1\) to \(0.3\).
    \item Set the prefix for all the files to be `example-'.
    \item Analysing 3 opponents.(Random will have multiple instances for different seeds.) 
\end{itemize}
\begin{figure}[ht]
    \centering
    \inputminted{python}{code_snippets/analysisExample.py}
    \caption{Code to create a sequence result to optimise best score for 3 opponents}\label{code:analysisExample}
\end{figure}

After it has run the output should be stored in the `./output' directory.
If the code fails to run there may be issues with this directory being created.
There should be multiple output files, each with one opponents evolution stages through the generations.

\paragraph{Cloud Notebook Setup} 
If you want to use a Cloud service, such as Azure Notebooks or AWS Sagemaker, the set up is similar to the above just executed differently.
Installing Anaconda is not needed, the environment has the required installs.
Using pip to install the Axelrod libraries and download the full analysis script can be done in an integrated web terminal or directly in a notebook.
Figure~\ref{code:jupyterExample} shows and example in azure, copy these in your top few cells of your jupyter notebook and it will work as required.

\begin{figure}[ht]
    \inputminted{python}{code_snippets/jupyterCells.py}
    \caption{Cells for creating the jupyter instance of a research environment}\label{code:jupyterExample}
\end{figure}

\section{Codebase Contributions}
Throughout the project I had split time between writing my own code and expanding the Axelrod Dojo codebase. 
In modern software development there is a commitment in the developer community to track and reuse as much as possible, typically using a version control system such as Git or Subversion.
The majority of open source community development is conducted on github\cite{GitHub}, the Axelrod libraries along with most other libraries I used are hosted here.

\subsection{Version Control}
During the project I created a `fork' of the axelrod dojo in order to add content to the open source community. 
This resulted in using Git to create a new `branch' on which to write my code before asking the owners of the repository to pull my work back into the core product using a `pull request'(PR).
The PR opened for my code\footnote{https://github.com/Axelrod-Python/axelrod-dojo/pull/45} resulted in changes to 457 lines of code and 14 files, adding classes and fixing bug that had been previously flagged.
Figure~\ref{fig:PR-open} shows the initial scope of the PR.
\begin{figure}[ht]
    \includegraphics[width=1.0\textwidth, center]{./img/vcs/PR-Open.png}
    \caption{Description and commits for PR on Github}\label{fig:PR-open}
\end{figure}
