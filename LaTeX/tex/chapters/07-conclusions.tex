% !TEX root = ../main.tex
\chapter{Practical Applications for Solution Sequences}\label{ch:conclusions}

\section{summary of approach}
This part should defiantly include the fact it was hard to get information about the opponents.
Classification of opponents was quite difficult.

\section{summary of analysis execution}
What were actually doing is finding a sequence that will manipulate our opponent into providing us the most number of cooperation moves we can subsequently defect against.

\section{Applications of results}
maybe how this works with respect to pathfinding, who we should select in a set of opponents if we only have to play a single opponent in the set.
If we can score an average of 3.8 against op1 or 3 against op2, we play op2 etc.

game shows to win money?
We want to win as much money as possible but we don't mind how much our opponent wins, as long as we get lots.

Failing that we are manipulating the opponent for cooperation moves we can cooperate against, then defection moves we can defect against, then defection moves we must cooperate for.

\section{Potential follow up work}
create the player to beat all players though opponent lookup; we just need to create a solvable opponent identifier. 
